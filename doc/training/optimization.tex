\documentclass[10pt,aspectratio=169]{beamer}

\setbeamertemplate{frametitle}{%
  \begin{beamercolorbox}[wd=\paperwidth, leftskip=1.2cm, sep=0pt, ht=1.0cm, dp=0.0cm]{frametitle}
    \vspace{-0.25cm}% adjust as needed
    \usebeamerfont{frametitle}\insertframetitle%
  \end{beamercolorbox}%
}

\usebackgroundtemplate{\includegraphics[height=\paperheight]{inl_pdf_template.pdf}}
\setbeamertemplate{navigation symbols}{}

\setbeamersize{text margin left=1em, text margin right=1em}

%------------- Beamer INL Above
\usepackage{amsmath, amssymb, graphicx}


\title{PDE-Constrained Optimization in MOOSE}
\author{CMDT}
\institute{Idaho National Lab}
\date{}

\begin{document}

% Title Slide
\begin{frame}
    \titlepage
\end{frame}

% Outline Slide
\begin{frame}{Outline}
	\vspace{0.5cm}
    \tableofcontents
\end{frame}

%-------------------------------------------------
\section{Introduction}

\begin{frame}{Overview of PDE-Based Optimization in MOOSE}
    \centering
\includegraphics[width=0.8\textwidth]{fig_optCycle.png}
\end{frame}

%-------------------------------------------------
\section{PDE-Constrained Optimization Concepts}

\begin{frame}{General Formulation}
    \begin{itemize}
        \item PDE constraint: 
        \[
          \mathcal{R}(\mathbf{u}, \mathbf{p}) = 0
        \]
        \item Objective function: 
        \[
          f(\mathbf{u}, \mathbf{p})
        \]
        \item Overall problem:
        \[
          \min_{\mathbf{p}} f(\mathbf{u}, \mathbf{p}) 
          \quad \text{subject to} \quad 
          \mathcal{R}(\mathbf{u}, \mathbf{p}) = 0
        \]
        \item Challenges:
        \begin{itemize}
            \item High dimensionality and nonlinearity.
            \item Coupled multi-physics PDEs.
        \end{itemize}
    \end{itemize}
\end{frame}



\begin{frame}{Adjoint Equation Derivation}
\small
\begin{enumerate}
    \item Total derivative of objective function:
      \[
        \frac{\mathrm{d}f(\mathbf{u},\mathbf{p})}{\mathrm{d}\mathbf{p}} = \frac{\partial f}{\partial \mathbf{u}}\frac{\partial \mathbf{u}}{\partial\mathbf{p}} + \frac{\partial f}{\partial \mathbf{p}}
      \]
  
      
    \item From the PDE constraint $\mathcal{R}(\mathbf{u},\mathbf{p})=\mathbf{0}$:
      \[
        \frac{\mathrm{d}\mathcal{R}(\mathbf{u},\mathbf{p})}{\mathrm{d}\mathbf{p}} = \frac{\partial \mathcal{R}}{\partial \mathbf{u}}\frac{\partial \mathbf{u}}{\partial\mathbf{p}} + \frac{\partial \mathcal{R}}{\partial \mathbf{p}} = 0
      \]
      
    \item Solving for the sensitivity matrix:
      \[
        \frac{\partial \mathbf{u}}{\partial\mathbf{p}} = -\left(\frac{\partial \mathcal{R}}{\partial \mathbf{u}}\right)^{-1}\frac{\partial \mathcal{R}}{\partial \mathbf{p}}
      \]
      
    \item Challenge: Computing $\frac{\partial \mathbf{u}}{\partial\mathbf{p}}$ directly requires solving one system per parameter
\end{enumerate}
\end{frame}

\begin{frame}{Adjoint Equation Derivation}
\small
\begin{enumerate}
    \setcounter{enumi}{4}
    \item Substituting sensitivity into objective gradient:
      \[
        \frac{\mathrm{d}f}{\mathrm{d}\mathbf{p}} = -\frac{\partial f}{\partial \mathbf{u}}\left(\frac{\partial \mathcal{R}}{\partial \mathbf{u}}\right)^{-1}\frac{\partial \mathcal{R}}{\partial \mathbf{p}} + \alpha \mathbf{p}
      \]
      
    \item Define adjoint variable $\boldsymbol{\lambda}$:
      \[
        \boldsymbol{\lambda}^T = \frac{\partial f}{\partial \mathbf{u}}\left(\frac{\partial \mathcal{R}}{\partial \mathbf{u}}\right)^{-1}
      \]
      
    \item Rearranging to get adjoint equation:
      \[
        \left(\frac{\partial \mathcal{R}}{\partial \mathbf{u}}\right)^T \boldsymbol{\lambda} = -\left(\frac{\partial f}{\partial \mathbf{u}}\right)^T
      \]
      
    \item Final gradient expression:
      \[
        \frac{\mathrm{d}f}{\mathrm{d}\mathbf{p}} = \boldsymbol{\lambda}^T \frac{\partial \mathcal{R}}{\partial \mathbf{p}} +  \frac{\partial f}{\partial p}
      \]
\end{enumerate}
\end{frame}

\begin{frame}{Adjoint Method: Key Components for Inverse Problems}
\begin{center}
\fbox{\begin{minipage}{0.85\textwidth}
\begin{center}
\textbf{Essential Components for Any Inverse Problem}
\end{center}
\begin{itemize}
    \item $\frac{\partial f}{\partial \mathbf{u}}$: Objective function sensitivity to state variables
    \item $\frac{\partial f}{\partial \mathbf{p}}$: Direct parameter dependency (often regularization)
    \item $\frac{\partial \mathcal{R}}{\partial \mathbf{p}}$: PDE residual sensitivity to parameters
\end{itemize}
 \[
        \frac{\mathrm{d}f}{\mathrm{d}\mathbf{p}} = \boldsymbol{\lambda}^T \frac{\partial \mathcal{R}}{\partial \mathbf{p}} +  \frac{\partial f}{\partial p}
      \]
\[
        \left(\frac{\partial \mathcal{R}}{\partial \mathbf{u}}\right)^T \boldsymbol{\lambda} = -\left(\frac{\partial f}{\partial \mathbf{u}}\right)^T
      \]

\end{minipage}}
\end{center}
\end{frame}

%-------------------------------------------------
\section{Steady-State Heat Conduction Inversion}
\begin{frame}{Forward Problem \& Objective Function}
\small
\begin{itemize}
    \item \textbf{General Forward PDE:}
      \[
        \mathcal{R}\bigl(T,\mathbf{p}\bigr) 
        = 0,
        \quad\text{where } 
        \mathcal{R}
        = \nabla \cdot \bigl(\kappa\nabla T\bigr) + g_b.
      \]
    \item \textbf{Objective Function (Data Misfit with Tikhonov Regularization)}:
      \[
        f\bigl(T,\mathbf{p}\bigr)
        = \frac12 \sum_{i=1}^{N} \bigl[T(x_i) - \widetilde{T}_i\bigr]^2
        \;+\; \frac{\alpha}{2} \|\mathbf{p}\|^2.
      \]
    \item \(\mathbf{p}\) are the design  parameters we want to estimate.
    \item \(\alpha\) is the regularization parameter controlling the trade-off between data fit and solution stability.
    \item We'll look at three specific parameterizations:
      \begin{enumerate}
        \item \(g_b(\mathbf{p})\) (heat source inversion),
        \item \(\kappa\nabla T \cdot \mathbf{n} = h(\mathbf{p})\,(T - T_\infty)\) (convective BC),
        \item \(\kappa(\mathbf{p})\) (material property inversion).
      \end{enumerate}
\end{itemize}
\end{frame}
%-------------------------------------------------
\begin{frame}{Derivative of the Data-Misfit Objective}
\small
\begin{itemize}

    \item The objective function with regularization:
      \[
        f(T,\mathbf{p}) = \frac12 \sum_{i=1}^N \bigl[T_i - \widetilde{T}_i\bigr]^2 + \frac{\alpha}{2} \|\mathbf{p}\|^2.
      \]
    \item Derivative with respect to parameters:
      \[
        \frac{\partial f}{\partial \mathbf{p}} = \alpha \mathbf{p}.
      \]
    \item Derivative of with respect to temperature:
      \[
        \frac{\partial f}{\partial T}(x) 
        = \sum_{i=1}^N \delta(x - x_i)\,\bigl[T(x_i) - \widetilde{T}_i\bigr].
      \]
 	\end{itemize}
\end{frame}

%-------------------------------------------------
\subsection{Case 1: Parameterized Heat Source}

\begin{frame}{Case 1: Parameterized Heat Source \(g_b(\mathbf{p})\)}
\small
\begin{itemize}
    \item \textbf{PDE:}
      \[
        \underbrace{
        \nabla \cdot \bigl(\kappa\nabla T\bigr)
        + g_b(\mathbf{p})
        }_{\mathcal{R}(T,\mathbf{p})}
        = 0
        \quad\text{in}\;\Omega.
      \]
    \item Here \(\kappa\) is fixed/known, but \(g_b\) depends on parameters \(\mathbf{p}\).
    \item \(\displaystyle
       \frac{\partial \mathcal{R}}{\partial \mathbf{p}}
       = \frac{\partial}{\partial \mathbf{p}}
         \bigl[g_b(\mathbf{p})\bigr],
      \)
      and \(\frac{\partial}{\partial \mathbf{p}}\bigl[\kappa\nabla T\bigr] = 0\).
    \item \textbf{Example parameter forms}:
      \begin{itemize}
         \item Linear: \(g_b(x) = p_0 + p_1\,x.\)
           \[
             \frac{\partial g_b}{\partial p_0} = 1, 
             \quad
             \frac{\partial g_b}{\partial p_1} = x.
           \]
         \item Gaussian-like or \(\delta\)-functions: 
           \(\;g_b(x) = \sum_{i}p_i\,\delta(x-x_i).\)
           \[
             \frac{\partial g_b}{\partial p_i}
             = \delta(x - x_i).
           \]
      \end{itemize}
    \item \(\boxed{\frac{\partial \mathcal{R}}{\partial \mathbf{p}}
           = 
           \int_{\Omega}
             \phi\,
             \frac{\partial g_b(\mathbf{p})}{\partial \mathbf{p}}
           \,d\Omega 
           \;\;\text{(in FEM form)}}\).
\end{itemize}
\end{frame}


%-------------------------------------------------
\subsection{Case 2: Parameterized Convective Boundary Condition}

\begin{frame}{Case 2: Parameterized Convection \(h(\mathbf{p})\)}
\small
\begin{itemize}
    \item \textbf{Forward PDE}: 
      \[
        \nabla \cdot \bigl(\kappa\nabla T\bigr) + g_b = 0
        \quad\text{in}\;\Omega,
      \]
      with 
      \[
        \bigl(\kappa\nabla T\bigr)\cdot \mathbf{n}
        = h(\mathbf{p})\,(T - T_\infty)
        \quad\text{on}\;\Gamma_R.
      \]
    \item \(\mathbf{p}\) enters the PDE through the \textbf{Robin boundary flux}.
    \item In a weak/FEM formulation, \(\frac{\partial \mathcal{R}}{\partial \mathbf{p}}\) arises from differentiating that boundary integral term \(\int_{\Gamma_R} \kappa \nabla T \cdot \mathbf{n}\,\phi - h(\mathbf{p})(T - T_\infty)\,\phi\).
    \[
       \frac{\partial \mathcal{R}}{\partial p_j}
       \;\propto\;
       \int_{\Gamma_R}
         \phi\,\frac{\partial}{\partial p_j} \Bigl[h(\mathbf{p})(T - T_\infty)\Bigr]
       \,d\Gamma.
    \]
    \item \textbf{If} \(h = p\) (a single scalar parameter):
      \(\displaystyle
       \frac{\partial}{\partial p} \bigl[h(T - T_\infty)\bigr]
       = (T - T_\infty).
      \)
    \item \(\boxed{\frac{\partial \mathcal{R}}{\partial \mathbf{p}}
           = 
           \int_{\Gamma_R}
             \phi\,
             \frac{\partial}{\partial \mathbf{p}}
             \bigl[h(\mathbf{p})(T - T_\infty)\bigr]
           \,d\Gamma}\).
\end{itemize}
\end{frame}


%-------------------------------------------------
\subsection{Case 3: Parameterized Material}

\begin{frame}{Case 3: Parameterized \(\kappa(\mathbf{p})\)}
\small
\begin{itemize}
    \item \textbf{PDE}:
      \[
        \nabla \cdot \Bigl(\kappa(\mathbf{p})\,\nabla T\Bigr)
        + g_b = 0
        \quad\text{in}\;\Omega.
      \]
    \item Now \(\mathbf{p}\) enters the \(\kappa\) term in the PDE’s \textit{operator}.
    \item \(\displaystyle
      \frac{\partial \mathcal{R}}{\partial \mathbf{p}}
      = 
      \frac{\partial}{\partial \mathbf{p}}
      \bigl[\nabla\cdot(\kappa(\mathbf{p})\nabla T)\bigr].
      \)
    \item \(\boxed{\frac{\partial \mathcal{R}}{\partial \mathbf{p}}
           = \nabla \cdot \Bigl(\frac{\partial \kappa(\mathbf{p})}{\partial \mathbf{p}}\nabla T\Bigr)
           \quad\text{(continuous form)}.}\)

    \item \textbf{FEM (discrete) form example}:
      \[
        \frac{\partial \mathcal{R}}{\partial p_j}
        =
        \int_{\Omega}
          \nabla \phi^\top
          \,\frac{\partial \kappa}{\partial p_j}
          \,\nabla T
        \; d\Omega.
      \]
    \item \(\textbf{Example: } 
       \kappa(T) = \alpha T^2 + \beta T 
       \Rightarrow
       \frac{\partial \kappa}{\partial \alpha} = T^2,\,
       \frac{\partial \kappa}{\partial \beta} = T.
    \)
\end{itemize}
\end{frame}

%-------------------------------------------------
%-------------------------------------------------
\section{Implementation in MOOSE}
\begin{frame}{Implementation in MOOSE: Overview}
    \begin{itemize}
        \item \textbf{Key Ingredients}:
        \begin{itemize}
            \item \textbf{AD} (Automatic Differentiation) for Jacobians and sensitivities.
            \item \textbf{MultiApps} and Transfers for outer optimization loops.
            \item \textbf{Executioner} and \textbf{Postprocessors} to evaluate objectives.
        \end{itemize}
        \item \textbf{Workflow}:
        \begin{enumerate}
            \item Define forward simulation (Kernels, BCs, Materials).
            \item Set up an objective function (Postprocessor/AuxVariable).
            \item Solve or approximate gradient (adjoint or FD).
            \item Update design variables (force, material property, density, etc.).
            \item Repeat until convergence.
        \end{enumerate}
    \end{itemize}
\end{frame}

\begin{frame}{Automatic Adjoint Computation in MOOSE}
    \begin{itemize}
        \item \textbf{Key Insight}: The Jacobian matrix from the forward problem can be reused for the adjoint!
        
        \item \textbf{Forward Problem}: $\mathcal{R}(\mathbf{u}, \mathbf{p}) = 0$
        \begin{itemize}
            \item Converged nonlinear solve gives us $\frac{\partial \mathcal{R}}{\partial \mathbf{u}}$ (the Jacobian)
        \end{itemize}
        
        \item \textbf{Adjoint Equation}: $\left(\frac{\partial \mathcal{R}}{\partial \mathbf{u}}\right)^T \boldsymbol{\lambda} = -\left(\frac{\partial f}{\partial \mathbf{u}}\right)^T$
        \begin{itemize}
            \item We can simply transpose the converged Jacobian and solve once
            \item Saves enormous development time (no manual adjoint derivation)
            \item Reuses existing nonlinear solution infrastructure
        \end{itemize}
        
      
    \end{itemize}
\end{frame}



\begin{frame}{Putting It All Together: Adjoint Workflow in MOOSE}
    \begin{itemize}
        \item \textbf{Full Adjoint-Based Optimization Workflow}:
        \begin{enumerate}
            \item Solve forward problem with AD-enabled physics
            \item Compute objective function (also using AD)
            \item Solve adjoint problem by transposing converged Jacobian
            \item Assemble gradient $\frac{df}{d\mathbf{p}}$ using adjoint solution
            \item Pass gradient to optimizer for parameter updates
        \end{enumerate}
        
        \item \textbf{MOOSE Components Used}:
        \begin{itemize}
            \item OptimizationApp: manages optimization loop
            \item Reporter system: transfers gradients and function values
            \item MultiApps: handles forward/adjoint solves
     
        \end{itemize}
        
        \item \textbf{Benefits}:
        \begin{itemize}
            \item Minimal code duplication between forward and adjoint
            \item Automatic differentiation reduces implementation errors
            \item Leverages existing MOOSE parallelism and solver capabilities
            \item Scales to complex multiphysics inverse problems
        \end{itemize}
    \end{itemize}
\end{frame}



%-------------------------------------------------
\section{Example Problems}
\subsection{Force Inversion}


\subsection{Nonlinear Material Parameterization}

\subsection{Transient Inverse Problem}


\subsection{Topology Optimization (SIMP)}



%-------------------------------------------------
\section{Conclusion}
\begin{frame}{Summary and Future Directions}
    \begin{itemize}
        \item \textbf{Summary}:
        \begin{itemize}
            \item MOOSE provides a robust framework for PDE-based optimization (inverse or topology).
            \item Adjoint methods are efficient for large-scale problems.
        \end{itemize}
        \item \textbf{Future Extensions}:
        \begin{itemize}
            \item Advanced multiphysics optimization (thermo-mechanical, fluid-structure).
            \item More sophisticated constraint handling (stress, manufacturing, etc.).
        \end{itemize}
    \end{itemize}
\end{frame}


\end{document}